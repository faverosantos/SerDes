%%%%%%%%%%%%%%%%%%%%%%%%%%%%%%%%%%%%%%%%%
% Lachaise Assignment
% LaTeX Template
% Version 1.0 (26/6/2018)
%
% This template originates from:
% http://www.LaTeXTemplates.com
%
% Authors:
% Marion Lachaise & François Févotte
% Vel (vel@LaTeXTemplates.com)
%
% License:
% CC BY-NC-SA 3.0 (http://creativecommons.org/licenses/by-nc-sa/3.0/)
% 
%%%%%%%%%%%%%%%%%%%%%%%%%%%%%%%%%%%%%%%%%

%----------------------------------------------------------------------------------------
%	PACKAGES AND OTHER DOCUMENT CONFIGURATIONS
%----------------------------------------------------------------------------------------

\documentclass{article}

\input{structure.tex} % Include the file specifying the document structure and custom commands

%----------------------------------------------------------------------------------------
%	ASSIGNMENT INFORMATION
%----------------------------------------------------------------------------------------

\title{SerDes analog circuits description} % Title of the assignment

\author{Fávero Santos\\ \texttt{favero.santos@gmail.com}} % Author name and email address

\date{Revision 00 of \today} % University, school and/or department name(s) and a date

%----------------------------------------------------------------------------------------

\begin{document}

\maketitle % Print the title

%----------------------------------------------------------------------------------------
%	INTRODUCTION
%----------------------------------------------------------------------------------------

\section{Introduction}

\begin{info}[\itshape Employed notation:] % Information block
	\begin{enumerate}[i]
		\item A DC only voltage/current is represented by an italic uppercase letter(s) followed by italic uppercase subscript letter(s).
		\subitem A DC voltage drop across the \textit{GATE} and \textit{SOURCE} of a transistor is represented by \textit{$V_{GS}$}.
		\item An AC only voltage/current is represented by an italic lowercase letter(s) followed by italic lowercase subscript letter(s).
		\subitem An AC voltage drop across the \textit{GATE} and \textit{SOURCE} of a transistor is represented by \textit{$v_{gs}$}.
		\item A combined DC and AC voltage/current is represented by an italic uppercase letter(s) followed by italic lowercase subscript letter(s).
		\subitem A combined DC and AC voltage drop across the \textit{GATE} and \textit{SOURCE} of a transistor is represented by \textit{$V_{gs}$}.
	\end{enumerate}
	An example of such notation is presented in Figure \ref{fig:NOTATIONR00}.
\end{info}

\begin{figure*}[!b]
	\centering
	\includegraphics[scale=0.5]{./imgs/NOTATIONR00.jpg}
	\caption{Examples of the employed notation in this work.}
	\label{fig:NOTATIONR00}
\end{figure*}

\section*{Analysis of an RC channel} % Unnumbered section

\subsection{Transfer function}

Consider the circuit

\begin{info}[\itshape Netlist in terms of time of an RC circuit] \newline
	R1 vin(t) vout(t) R \newline
	C1 vout(t) gnd!(t) C
\end{info}

in which the a current i(t) is sourced in the vin(t) node and drained at gnd!

Applying the Laplace transform, the circuit becomes

\begin{info}[\itshape Netlist in terms of frequency of an RC circuit] \newline
	R1 vin(s) vout(s) R \newline
	C1 vout(s) gnd!(t) 1/sC
\end{info}

At node vout(s):

\centerline{$ v_{out}(s) = i(s) \cdot 1/(s \cdot C)
	$}
\vspace{\baselineskip}

\centerline{$ v_{in}(s) - v_{out}(s) = i(s) \cdot R
	$}
\vspace{\baselineskip}

\centerline{$ v_{in}(s) - i(s) \cdot 1/(s \cdot C) = i(s) \cdot R
	$}
\vspace{\baselineskip}

\centerline{$ v_{in}(s) = i(s) \cdot (R + 1/(s \cdot C))
	$}
\vspace{\baselineskip}

Thus

\centerline{$ \frac{v_{out}(s)}{v_{in}(s)} = H(s) = \frac{i(s) \cdot 1/(s \cdot C)}{i(s) \cdot (R + 1/(s \cdot C))}
	$}
\vspace{\baselineskip}

\begin{question}[\itshape Transfer function for an RC circuit]
\centerline{$ H(s) = \frac{\omega_{a}}{s + \omega_{a}}
	$}
\end{question}

Where $\tau = R \cdot C $ and $\omega_{a} = 1/ \tau$.

\subsection{Impulse response}

Applying the inverse Laplace transform in the transfer function, it is obtained:

\begin{question}[\itshape Impulse response for an RC circuit]
\centerline{$ h(t) = \omega_{a} \cdot \exp(-t * \omega_{a})
	$}
\end{question}

for t > 0.

\subsection{Magnitude of the transfer function}

\centerline{$ |H(s)| = |\frac{\omega_{a}}{s + \omega_{a}}|
	$}
\vspace{\baselineskip}

\begin{question}[\itshape Magnitude of H(s)]
\centerline{$ |H(s)| = \frac{\omega_{a}}{\sqrt{w^2 + \omega_{a}^2}}
	$}
\end{question}

\subsection{Phase of the transfer function}

\centerline{$ \theta(s) = \arctan{(\frac{\Im{H(s)}}{\Re{H(s)}})}
	$}
\vspace{\baselineskip}

\centerline{$ H(s) = \frac{\omega_{a}}{i\cdot \omega + \omega_{a}} = \frac{\omega_{a} \cdot (-i \cdot \omega + \omega_{a})}{(i \cdot \omega + \omega_{a}) \cdot (-i \cdot \omega + \omega_{a})}
	$}
\vspace{\baselineskip}

\centerline{$ H(s) = \frac{\omega_{a}^2 - i \cdot \omega \cdot \omega_{a}}{\omega^2 + \omega_{a}^2}
	$}
\vspace{\baselineskip}

\centerline{$ \Re{H(s)} = \frac{\omega_{a}^2}{\omega^2 + \omega_{a}^2}
	$}
\vspace{\baselineskip}

\centerline{$ \Im{H(s)} = \frac{-\omega \cdot \omega_{a}}{\omega^2 + \omega_{a}^2}
	$}
\vspace{\baselineskip}

\begin{question}[\itshape Phase of H(s)]
\centerline{$ \theta(s) = \arctan{(- \omega / \omega_{a})}
	$}
\end{question}

\subsection{Quantization of the model}

Considering that $\omega_{a} = \omega_{d} / FS$, where FS is the sampling frequency, the quantized model is:

\begin{question}[\itshape Transfer function for an RC circuit]
	\centerline{$ H[m] = \frac{\omega_{d} / FS}{s + \omega_{d} / FS}
		$}
\end{question}
Where m is a discrete frequency vector.

\begin{question}[\itshape Impulse response for an RC circuit]
	\centerline{$ h[n] = (\omega_{d} / FS) \cdot \exp(-t * \omega_{d} / FS)
		$}
\end{question}
Where n is a discrete time vector.

\begin{question}[\itshape Magnitude of H(s)]
	\centerline{$ |H[m]| = \frac{\omega_{d} / FS}{\sqrt{w^2 + \omega_{d} / FS^2}}
		$}
\end{question}

\begin{question}[\itshape Phase of H(s)]
	\centerline{$ \theta(s) = \arctan{(- \omega / \omega_{d} / FS)}
		$}
\end{question}

\section{Example code}

An example code for such study can be found at: https://github.com/faverosantos/SerDes

%----------------------------------------------------------------------------------------

% References
\begin{thebibliography}{9}
	%\bibitem{techgurukula} 
	%\url{https://www.youtube.com/watch?v=bSlPaSw8pDw}
	
	%\bibitem{pornpromlikit}
	%file:///D:/Doutorado/2019\_Qualificação/Papers/Pornpromlikit\_2010.pdf
	
	%\bibitem{taylor} 
	%\url{https://math.libretexts.org/Bookshelves/Calculus/Supplemental_Modules_(Calculus)/Multivariable_Calculus/3\%3A_Topics_in_Partial_Derivatives/Taylor__Polynomials_of_Functions_of_Two_Variables}
	
	%\bibitem{sedra} 
	%Sedra, A. S. and Smith, K. C in "Microelectronics" 4th edition.	
\end{thebibliography}
%----------------------------------------------------------------------------------------

\end{document}
