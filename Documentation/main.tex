%%%%%%%%%%%%%%%%%%%%%%%%%%%%%%%%%%%%%%%%%
% Lachaise Assignment
% LaTeX Template
% Version 1.0 (26/6/2018)
%
% This template originates from:
% http://www.LaTeXTemplates.com
%
% Authors:
% Marion Lachaise & François Févotte
% Vel (vel@LaTeXTemplates.com)
%
% License:
% CC BY-NC-SA 3.0 (http://creativecommons.org/licenses/by-nc-sa/3.0/)
% 
%%%%%%%%%%%%%%%%%%%%%%%%%%%%%%%%%%%%%%%%%

%----------------------------------------------------------------------------------------
%	PACKAGES AND OTHER DOCUMENT CONFIGURATIONS
%----------------------------------------------------------------------------------------

\documentclass{article}

\input{structure.tex} % Include the file specifying the document structure and custom commands

%----------------------------------------------------------------------------------------
%	ASSIGNMENT INFORMATION
%----------------------------------------------------------------------------------------

\title{SerDes analog circuits description} % Title of the assignment

\author{Fávero Santos\\ \texttt{favero.santos@gmail.com}} % Author name and email address

\date{Revision 00 of \today} % University, school and/or department name(s) and a date

%----------------------------------------------------------------------------------------

\begin{document}

\maketitle % Print the title

%----------------------------------------------------------------------------------------
%	INTRODUCTION
%----------------------------------------------------------------------------------------

\section*{Introduction}

The netlists presented in this article can be seen as circuits if a netlist-to-schematic conversor (such as NetlistViewer - https://sourceforge.net/p/netlistviewer/wiki/Home/) is employed.

\begin{info}[\itshape Employed notation:] % Information block
	\begin{enumerate}[i]
		\item A DC only voltage/current is represented by an italic uppercase letter(s) followed by italic uppercase subscript letter(s).
		\subitem A DC voltage drop across the \textit{GATE} and \textit{SOURCE} of a transistor is represented by \textit{$V_{GS}$}.
		\item An AC only voltage/current is represented by an italic lowercase letter(s) followed by italic lowercase subscript letter(s).
		\subitem An AC voltage drop across the \textit{GATE} and \textit{SOURCE} of a transistor is represented by \textit{$v_{gs}$}.
		\item A combined DC and AC voltage/current is represented by an italic uppercase letter(s) followed by italic lowercase subscript letter(s).
		\subitem A combined DC and AC voltage drop across the \textit{GATE} and \textit{SOURCE} of a transistor is represented by \textit{$V_{gs}$}.
	\end{enumerate}
	An example of such notation is presented in Figure \ref{fig:NOTATIONR00}.
\end{info}

\begin{figure*}[!b]
	\centering
	\includegraphics[scale=0.5]{./imgs/NOTATIONR00.jpg}
	\caption{Examples of the employed notation in this work.}
	\label{fig:NOTATIONR00}
\end{figure*}

\section{Analysis of an RC channel} % Unnumbered section

\subsection{Transfer function}

Consider the circuit

\begin{file}[\itshape Netlist of an RC circuit]
	
.SUBCKT RCchannel vin vout 	\newline
R1 vin vout R				\newline
C1 vout gnd! C				\newline
.ENDS
\end{file}

At node vout(s):

\centerline{$ v_{out}(s) = i(s) \cdot 1/(s \cdot C)
	$}
\vspace{\baselineskip}

\centerline{$ v_{in}(s) - v_{out}(s) = i(s) \cdot R
	$}
\vspace{\baselineskip}

\centerline{$ v_{in}(s) - i(s) \cdot 1/(s \cdot C) = i(s) \cdot R
	$}
\vspace{\baselineskip}

\centerline{$ v_{in}(s) = i(s) \cdot (R + 1/(s \cdot C))
	$}
\vspace{\baselineskip}

Thus

\centerline{$ \frac{v_{out}(s)}{v_{in}(s)} = H(s) = \frac{i(s) \cdot 1/(s \cdot C)}{i(s) \cdot (R + 1/(s \cdot C))}
	$}
\vspace{\baselineskip}

\begin{question}[\itshape Transfer function for an RC circuit]
\centerline{$ H(s) = \frac{\omega_{a}}{s + \omega_{a}}
	$}
\end{question}

Where $\tau = R \cdot C $ and $\omega_{a} = 1/ \tau$.

\subsection{Impulse response}

Applying the inverse Laplace transform in the transfer function, it is obtained:

\begin{question}[\itshape Impulse response for an RC circuit]
\centerline{$ h(t) = \omega_{a} \cdot \exp(-t * \omega_{a})
	$}
\end{question}

for t > 0.

\subsection{Magnitude of the transfer function}

\centerline{$ |H(s)| = |\frac{\omega_{a}}{s + \omega_{a}}|
	$}
\vspace{\baselineskip}

\begin{question}[\itshape Magnitude of H(s)]
\centerline{$ |H(s)| = \frac{\omega_{a}}{\sqrt{w^2 + \omega_{a}^2}}
	$}
\end{question}

\subsection{Phase of the transfer function}

\centerline{$ \theta(w) = \arctan{(\frac{\Im{H(w)}}{\Re{H(w)}})}
	$}
\vspace{\baselineskip}

\centerline{$ H(w) = \frac{\omega_{a}}{i\cdot \omega + \omega_{a}} = \frac{\omega_{a} \cdot (-i \cdot \omega + \omega_{a})}{(i \cdot \omega + \omega_{a}) \cdot (-i \cdot \omega + \omega_{a})}
	$}
\vspace{\baselineskip}

\centerline{$ H(w) = \frac{\omega_{a}^2 - i \cdot \omega \cdot \omega_{a}}{\omega^2 + \omega_{a}^2}
	$}
\vspace{\baselineskip}

\centerline{$ \Re{H(w)} = \frac{\omega_{a}^2}{\omega^2 + \omega_{a}^2}
	$}
\vspace{\baselineskip}

\centerline{$ \Im{H(w)} = \frac{-\omega \cdot \omega_{a}}{\omega^2 + \omega_{a}^2}
	$}
\vspace{\baselineskip}

\begin{question}[\itshape Phase of H(w)]
\centerline{$ \theta(w) = \arctan{(- \omega / \omega_{a})}
	$}
\end{question}

\subsection{Quantization of the model}

Considering that $\omega_{a} = \omega_{d} / FS$, where FS is the sampling frequency, the quantized model is:

\begin{question}[\itshape Transfer function for an RC circuit]
	\centerline{$ H[m] = \frac{\omega_{d} / FS}{s + \omega_{d} / FS}
		$}
\end{question}
Where m is a discrete frequency vector.

\begin{question}[\itshape Impulse response for an RC circuit]
	\centerline{$ h[n] = (\omega_{d} / FS) \cdot \exp(-n * (\omega_{d} / FS))
		$}
\end{question}
Where n is a discrete time vector.

\begin{question}[\itshape Magnitude of H(w)]
	\centerline{$ |H[m]| = \frac{\omega_{d} / FS}{\sqrt{w^2 + (\omega_{d} / FS)^2}}
		$}
\end{question}

\begin{question}[\itshape Phase of H(w)]
	\centerline{$ \theta[m] = \arctan{(- \omega /(\omega_{d} / FS))}
		$}
\end{question}

\section{Analysis of a constant time linear equalizer}

\subsection{Transfer function}

Consider the circuit

\begin{file}[\itshape Netlist of a CTLE]

.SUBCKT CTLE vinp vinn voutp voutn gnd! \newline
R1 gnd! voutp RD						\newline
M1 voutp vinp vs1 0 M1					\newline
I1 vs1 gnd! I1							\newline
R2 gnd! voutn RD						\newline
M2 voutn vinn vs2 0 M2					\newline
I2 vs2 gnd! I1							\newline
R3 vs1 vs2 RS							\newline
C1 vs1 vs2 CS							\newline
.ENDS
\end{file}

Its transfer function is:

\begin{question}[\itshape Transfer function for a CTLE circuit]
	\centerline{$ H(s) = \frac{-g_m \cdot R_D \cdot (s + \omega_a)}{(s + \omega_a \cdot (\tau + h))}
		$}
\end{question}

Where $\tau = RS \cdot CS$, $\omega_a = 1/\tau$, and $h = g_m \cdot RS / 2$.

\subsection{Impulse response}

\begin{question}[\itshape Transfer function for a CTLE circuit]
	\centerline{$ h(t) = -g_m \cdot R_D \cdot \omega_a \cdot (1 - h - \tau) \cdot exp(-t \cdot \omega_a \cdot (h + \tau) )
		$}
\end{question}

For t > 0.

\subsection{Magnitude of the transfer function}

\centerline{$ |H(w)| = \frac{g_m \cdot R_D \cdot \sqrt{w_a^2 + w^2}}{\sqrt{(w_a \cdot (\tau + h))^2 + \omega^2}}
	$}
\vspace{\baselineskip}

\subsection{Phase of the transfer function}

\centerline{$ \Re{H(w)} = \frac{g_m \cdot R_D \cdot (\omega^2 + \omega_a^2 \cdot (\tau + h))}{\omega^2 + \omega_a^2 \cdot (\tau + h)^2}
	$}
\vspace{\baselineskip}

\centerline{$ \Im{H(w)} = \frac{gm \cdot R_D \cdot \omega \cdot \omega_a \cdot (\tau + h - 1)}{\omega^2 + \omega_a^2 \cdot (\tau + h)}
	$}
\vspace{\baselineskip}

\begin{question}[\itshape Phase of H(w)]
	\centerline{$ \theta(w) = \arctan{\frac{\omega \cdot \omega_a \cdot (\tau + h - 1)}{\omega^2 + \omega_a^2 \cdot (\tau + h)^2}}
		$}
\end{question}

\subsection{Quantization of the model}

Considering that $\omega_{a} = \omega_{d} / FS$, where FS is the sampling frequency, the quantized model is:

\begin{question}[\itshape Transfer function for a CTLE circuit]
	\centerline{$ H[m] = \frac{-g_m \cdot R_D \cdot (s + (\omega_d / FS))}{(s + (\omega_d / FS) \cdot (\tau + h))}
		$}
\end{question}

\begin{question}[\itshape Transfer function for a CTLE circuit]
	\centerline{$ h[n] = -g_m \cdot R_D \cdot (\omega_d / FS) \cdot (1 - h - \tau) \cdot exp(-n \cdot (\omega_d / FS) \cdot (h + \tau) )
		$}
\end{question}

\begin{question}[\itshape Magnitude of H(w)]
\centerline{$ |H[m]| = \frac{g_m \cdot R_D \cdot \sqrt{(\omega_d / FS)^2 + \omega^2}}{\sqrt{((\omega_d / FS) \cdot (\tau + h))^2 + \omega^2}}
	$}
\vspace{\baselineskip}
\end{question}

\begin{question}[\itshape Phase of H(w)]
	\centerline{$ \theta[m] = \arctan{\frac{\omega \cdot (\omega_d / FS) \cdot (\tau + h - 1)}{\omega^2 + (\omega_d / FS)^2 \cdot (\tau + h)^2}}
	$}
\end{question}

\section{Example code}

An example code for such study can be found at: https://github.com/faverosantos/SerDes

%----------------------------------------------------------------------------------------

% References
\begin{thebibliography}{9}
	%\bibitem{techgurukula} 
	%\url{https://www.youtube.com/watch?v=bSlPaSw8pDw}
	
	%\bibitem{pornpromlikit}
	%file:///D:/Doutorado/2019\_Qualificação/Papers/Pornpromlikit\_2010.pdf
	
	%\bibitem{taylor} 
	%\url{https://math.libretexts.org/Bookshelves/Calculus/Supplemental_Modules_(Calculus)/Multivariable_Calculus/3\%3A_Topics_in_Partial_Derivatives/Taylor__Polynomials_of_Functions_of_Two_Variables}
	
	%\bibitem{sedra} 
	%Sedra, A. S. and Smith, K. C in "Microelectronics" 4th edition.	
\end{thebibliography}
%----------------------------------------------------------------------------------------

\end{document}
